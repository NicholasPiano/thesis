%*******************************************************************************
%****************************** Second Chapter *********************************
%*******************************************************************************

\chapter{The microfludics environment}

\ifpdf
    \graphicspath{{Chapter2/Figs/Raster/}{Chapter2/Figs/PDF/}{Chapter2/Figs/}}
\else
    \graphicspath{{Chapter2/Figs/Vector/}{Chapter2/Figs/}}
\fi

%********************************** %First Section  **************************************
\section{Description of the environment}

The microfludics environment used for this study consisted of a microchannel framework printed using soft lithography on a PDMS chip. PDMS, or Polydimethylsiloxane, is a a silicone gel that can be molded and used to create microscopic structures, such as channels [ref]. This microchannel technique is widely used in the biomedical industry to mimic body tissue for complex in vitro [italic] studies. In this case, the channel was set up to model a human blood vessel. The diagram in Figure [ref] below shows part of the channel framework with two PDMS pillars on either side of a gap. On one side of the gap, a liquid medium meant to simulate blood plasma is pumped in where it can remain static, or be used to simulate the flow inside a blood vessel. On the other side of the gap, collagen gel, used to simulate the extra-cellular matrix surrounding a blood vessel, provides an anchor point for the endothelial cells, or cells found in the blood vessel wall, to attach to. These endothelial cells are added to the environment prior to the experiment. The experiment is monitored when cancer cells that are marked with a fluorescent GFP, or Green Fluorescent Protein, are added to the environment in the medium channel, from which they can cross the endothelial cell barrier into the collagen gel, as they might in vivo [italic] during metastasis [ref].

The entire channel is approximately 100 microns high [ref], or about 5 times the height of a typical cancer cell used in the experiments [ref]. The width of the gap containing the endothelial cell barrier is between 100-200 microns [ref]. A typical blood vessel would be tubular, but this setup, while possible [ref], would be difficult to study and to image with a microscope. Opting instead for the simplified setup, the vertical wall of cells can be observed much more easily and still be used to provide information on how the cancer cells cross the barrier.

\section{Purpose and the need for segmentation}

The purpose of this environment is to determine how cancer cells cross the endothelium (cell barrier) during metastasis. Metastasis is a poorly understood process in cancer biology, and this type of system could help understand it by providing accurate data on the morphodynamics of the cells as they move in the environment.

\section{Limitations and 3D structure}
