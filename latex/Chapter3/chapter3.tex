%*******************************************************************************
%****************************** Third Chapter *********************************
%*******************************************************************************

\chapter{Background}

\ifpdf
    \graphicspath{{Chapter3/Figs/Raster/}{Chapter3/Figs/PDF/}{Chapter3/Figs/}}
\else
    \graphicspath{{Chapter3/Figs/Vector/}{Chapter3/Figs/}}
\fi

%********************************** %First Section  **************************************
\section{Introduction}

A comprehensive background on the key concepts that the method presented in this thesis is based on is essential for understanding its context and significance. Any physical or mathematical principles that might affect the performance of the method should also be understood. The previous chapter outlined some of the problems faced when attempting to segment cancer cells given a 3D live cell environment. These included autofocus fluctuations of the microscope and limits on light absorption by cells. This chapter gives some detail on topics including the GFP distribution within the cells, which plays a key role in the pre-processing method, basic image processing and segmentation and how these techniques are used in Cellprofiler and other segmentation software packages, and finally an introduction to the brightfield method on which the current method is based.

\section{Cell microscopy, optical structure, and GFP distributions}

Under a microscope, the visible parts of a cell, the optical properties of edges, and colours observed depend on the materials that it contains. A typical human cell is colourless and transparent [ref], but when light passes tangentially through the cell wall, it can be blocked and give the appearance of a dark edge. This leads to a projected (2D) shape under the microscope. This shape is a closed curve and thus has an interior (the material within the cell) and an exterior (the background). This is an assumption made throughout the segmentation process, since it is not possible for the cell to be an open shape. This is useful for segmentation, because there is a natural boundary that allows the recognition to be restricted to closed curves made up of dark edges.

Cells can move in response to chemical stimuli [ref]. In order to move in their environment, a cell can change tension in its outer wall and form protrusions that stretch in the direction of a stimulus [ref]. This can allow the cell to propel itself and follow chemical gradients. These protrusions, if they can be measured, provide useful information about the behaviour of the cell in response to any agents that have been added to the system as part of the experiment. The microfluidics environment was designed with the addition of chemical agents in mind. Even without added stimuli, the purpose of the experiment is to be able to observe the cells moving and allow for their shapes to be measured. Accurate recognition of cell protrusions is a key aim of this study. Without it, cell behaviour cannot be judged in changing conditions.

Cells can be marked by injecting a variety of fluorescent proteins into different parts of their internal structure such as the internal cytosol, the cell membrane, and the nucleus. A common protein used is known as GFP, or Green Fluorescent Protein. It is usually expressed in the cytosol, a clear fluid surrounding the nucleus, occupying the bulk of the cell, but not the membrane or the nucleus [ref]. When a 3D reconstruction is created using a confocal microscope, this distribution of GFP will appear as a single bulk shape with a gap at the location of the nucleus. It notably does not contain information about the extremities of the cell such as protrusions. It can also be expressed in the cell membrane, leading to a hollow 3D shape that shows clearly the extremities of the cell.

[IMG distributions of GFP]

\section{Image processing and segmentation}

One of the most important features that can be extracted from an image is the location of edges. They delimit objects and provide evidence of texture within an object. An edge is defined as an intensity discontinuity in one dimension. Crossing the edge, the intensity can be plotted and represented as a mathematical relationship between the intensity and the path taken. Movement across a thin dark edge will show a decrease in intensity followed by a subsequent increase once the edge is passed. This mathematical model allows edges in a set of images to found quickly and reliably if it is general enough. The ability to rapidly find edges in an image is a key tool in a segmentation program, since this is the most certain way of locating objects. An example of a widely used edge detection method is called the Canny filter. Proposed by John Canny in his 1986 paper, A Computational Approach To Edge Detection [italic]. It works by first smoothing the image to reduce noise, finding the intensity gradients of the image (edges will show a high intensity gradient), suppressing points with less than the maximum gradient, along with applying an absolute intensity threshold, and finally joining potential edge segments by removing those not connected to strong edge segments. Other features such as corners can be modelled as superpositions of the simple edges in different orientations.

[IMG the Canny filter applied to current data]

Another tool for recognition is the blob detector. A ``blob" is a technical term meaning an contiguous (interconnected) region of uniform (similar) colour. This could indicate a solid object or a region of background. Depending on the type of imaging, blobs that represent objects can either be light or dark, but both can be found by modelling the region as a 2D intensity discontinuity. This can be represented mathematically as a 2D Gaussian of arbitrary, but fixed, radius. Such a modelling function can then be convolved with the image in a similar manner to simple edges. Parts of the image that contain regions of colour of size proportional to the radius of the Gaussian will give a response to the convolution. The areas with the highest response are the most likely candidates for objects of the specified size.

[IMG Blobs Gaussian convolution]

Once blobs and edges have been found, it is possible that in a given cellular environment, cells will be found packed closely in dense clusters. In order to be differentiated from each other, boundaries between the cells must be drawn based on the image properties surrounding them. Often, cells are separated by sharp discontinuities such as dark edges, but they may be so close that optically, their intensity curves appear to transition smoothly from one cell to another. In this case, a variant of the watershed method is useful for separating them [ref]. like many image processing techniques, the watershed method models the intensity map of the image as a terrain, where the contour height at a point is proportional to its intensity. If one imagines the terrain as being slowly filled with a certain amount of water, highly divergent regions such as cell representations will stand out like hills and the point where each ``hill" meets the water is the boundary of that cell. If the water is raised high enough, even smoothly transitioning boundaries between cells will appear to have a dividing line between them. This can be used to segment closely packed objects using their intensity peaks alone.

[IMG Watershed example]

\section{CellProfiler and segmentation software}

The last section details some of the methods used by Cellprofiler and other software packages, such as several ImageJ plugins, to segment cells. When Cellprofiler is provided with an image, it will first look for blobs of colour in the image. Dependent on the background, it will look for bright or dark blobs as potential candidates for objects. The edges close to these blobs are kept as potential object edges. The blobs of colour are then separated using a watershed method. Their contours are matched with the edge image and the space is filled in to form an object ``mask". The mask is a binary representation the object of interest. Every point in the mask is considered part of the object, and all points outside are considered part of the background. The watershed method allows adjacent masks to be separated. Every task that Cellprofiler performs on an image such as edge detection or blob detection is typically contained within a module that stores information about operations done to the image. These modules can be chained to compound effects and allow very specific workflows to suit all types of images. It can also combine images from several channels using basic mathematical operations.

[IMG masks]

Cellprofiler has a particularly useful module called ``Secondary Objects". Rather than searching the image for objects from scratch, it can use information about previously found objects to search in their vicinity. If object centres are represented as points, Secondary Objects can search around the points and use information about the vicinity of the point to match similar patches of colour or edge profiles. This operation is very adaptible, and does not have a size preference, filling whatever space is available. This is highly valued since one of the driving philosophies of this work is that recognition should not be biased towards a particular size or shape. A disadvantage of this approach is a recognition that expands without limit. If there is no discernible boundary, there is no reason for the recognition to stop expanding into the background if it is of similar intensity. To clarify, if the background of the image and the interior of the cell have similar colours, and the edges of the cell are incomplete or have an inconsistent profile, the recognition could add parts of the background to the final mask of the cell. This can affect the method used, but eventually adding 3D information to this 2D method can solve the problem.

[IMG secondary objects expanding too far]

Another tool for segmentation is the Blow/Lasso tool in ImageJ. This treats the image as a ``terrain", where the ``height" of a pixel is proportional to its itensity. If an algorithm follows a path through the image, moving from one pixel to another can be assigned a cost based on the relationship between the intensities. An algorithm could be set up such that moving into dark edges could be very cheap, but moving out of them could be expensive, causing the path to follow a chain of edges closely. In a similar fashion, two points can be specified and the cost between them calculated [ref]. If one point is specified as the centre from which to search, the locus of points that share the same cost from the centre can yield a closed shape with dark edges such as a cell. While not reliable without tuning, the cost function can be made more specific to suit the application. Such a function can also be made to take account of multiple channels. Depending on how it is implemented, considering a central point can bias the shape towards a circular locus.

[IMG blow/lasso example]

A notable disadvantage of Cellprofiler is the lack of comprehensive support for 3D environments and sets of images with 3D relationships. Although it can be made to behave in this way, images must be processed on an individual basis or in basic groupings, limiting the complexity of 3D operations. This makes true 3D segmentation difficult or impossible. Nevertheless, Cellprofiler has a wide array of very powerful 2D operations. Thus an ideal solution is a method that is able to cast a 3D dataset in a 2D context in order to take advantage of 2D methods while preserving 3D data.

The final data output from Cellprofiler is a list of object masks along with some analysis of their shape such as their eccentricity or orientation (from a bounding ellipse). Cellprofiler does allow the segmentation of connected objects, such as protrusions, and can provide their measurements, but this can be unreliable since the properties of protrusions are hard to specify. A larger problem with Cellprofiler is the inability to adapt to inconsistencies. For example, the edges of a cell protrusion can have a very different appearance to the main body of the cell as the cell walls grow thinner. While the edge of a protrusion maybe remain visible towards its apex, the difference in edge properties might cause Cellprofiler to conclude that they are unrelated edges, and lead to a protrusion being rejected and not represented in the final data. This unreliability with regards to ``tertiary object" like protrusions is insufficient for the accuracy required by this project. The method in Chapter [ref] will show how 3D information can enhance the recognition of tertiary objects.

\section{Cell tracking}

Cell tracking is any method of associating recognised objects in different images gathered at different times and assigning them same identity. This can allow properties of the cells such as area and position to be plotted over time and for trends to be observed. In a low density packing of cells, this can be as simple as finding the distance between objects in different frames and assigning the next iteration to be the object with the smallest distance from the starting point. In this study, connected iterations of a single physical object are referred to as ``instances" as in ``cell instance". This is an important distinction, since a cell objects cannot be said to have an area or velocity since these properties change with time. Assigning an area to a particular cell at a point in time is equivalent to assigning the area to the cell instance. Thus a single cell can have many cell instances.

A popular tracking method is the LAP, or ``Linear Assignment Problem", framework tracking algorithm. Once two lists of objects and their positions in subsequent images are found, they can be connected and ranked by a likelihood of correspondence. This is done by first constructing and solving a NxN [maths] matrix of correspondence parameters where N is the number of objects considered. A similar matrix is constructed to represent the probabilities of cells merging or splitting. The final step reconciles the two images into a list of corresponding cell instances. This process can then be repeated for subsequent frames. This algorithm is employed by Cellprofiler as a module, so it can be applied to objects found through recognition. Other properties of the cells such as changing shape and size can contribute to the likelihood of correspondence. LAP also helps to associate cells that disappear and reappear, as well as cells that merge or split.

The method yields four important numbers, namely ``LostObjectCount", ``NewObjectCount", ``SplitObjectCount", and ``MergedObjectCount". A measure of the reliability of the tracking is given by the consistency of these numbers. For example, if NewObjectCount continues to increase throughout the time series, it is likely that objects that should be connected are being rejected and treated as new objects. This can happen if objects move a long distance, such as many multiples of their length, between frames. If the ratio of distance moved to time between frames is too high, the tracking algorithm may be too unreliable for use. In this case, manual tracking may be necessary to provide accurate information about cells.

[IMG increasing NewObjectCount]

\section{The Selinummi brightfield profile method}

In their 2009 paper, Bright Field Microscopy as an Alternative to Whole Cell Fluorescence in Automated Analysis of Macrophage Images [italic], Selinummi et al. describe a method of using 3D brightfield information to aid segmentation instead of relying on GFP imaging of cells. The images for their study were gathered in a similar way to this study. A confocal microscope was used to scan a 3D environment gathering images in brightfield and GFP channels. Gathering data with a confocal microscope is slow and expensive, so if adequate cell data can be produced using a single channel, this can save time. They proposed using the amplitude of the variation of the brightfield intensity in the Z dimension as an indicator of the presence of an object at a particular XY location. This exploits the behaviour of the brightfield representations of objects as they move in and out of focus. As outlined in Chapter [ref], the brightfield does not store 3D information, but changes with the focal plane of the environment. An area of background will not vary across focal planes, but a location or single pixel containing an object will vary with focal plane.

The basis of the method is an inspiration for the current project. The vertical distribution of the brightfield intensities at a particular XY location, known as a ``profile" in this study, shows how the features vary in intensity across Z.

[IMG Brightfield profile]

Their results were promising for the data they gathered, but suffered when applied to the data from the current experiment. To test the method, they compared cell segmentation of the modified brightfield images to the GFP images of the same cells. They used the GFP segmentation as the ``ground truth" of their testing. Ground truth is assumed to be the best representation of the object. This is a failing of the study since there are many details omitted by the GFP, especially considering the variety of configurations cellular GFP can take, as discussed in Section [ref], earlier in this chapter. Segmentation was compared pixel by pixel using the ``FScore", or a compound ratio of the true and false positive scores.

There are several disadvantages of the Selinummi method, especially when applied to the current data. Firstly, the method was originally applied to a single cell environment with no visible materials other than cells. in contrast, the current environment is primarily composed of PDMS plastic. It is also a multicellular environment. The images thus contain many other objects in the brightfield that are not desired as part of the dataset, but they still have similar edges and strong variations in their intensity profiles. Regions of the image where any kind of object is found will appear bright. This can hinder accurate segmentation since unwanted objects are highlighted. Secondly, due to the brightfield variation in the images, strongly varying pixels can occur outside the true edge of the cell, causing these regions to be highlighted, and giving the cell a bright halo. This halo is picked up in the recognition and contributes to the measured area of the cell. This can cause an over-estimation of the cell area and a false representation of the cell's shape. These problems must be addressed for a variant of this method to be useful.
