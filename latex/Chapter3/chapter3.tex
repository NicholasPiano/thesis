%*******************************************************************************
%****************************** Third Chapter *********************************
%*******************************************************************************

\chapter{Background}

\ifpdf
    \graphicspath{{Chapter3/Figs/Raster/}{Chapter3/Figs/PDF/}{Chapter3/Figs/}}
\else
    \graphicspath{{Chapter3/Figs/Vector/}{Chapter3/Figs/}}
\fi

%********************************** %First Section  **************************************
\section{Introduction}

A comprehensive background on the key concepts that the method presented in this thesis is based on is essential for understanding its context and significance. The previous chapter outlined some of the problems faced when attempting to segment cancer cells given a 3D live cell environment. These included autofocus fluctuations of the microscope and limits on light absorption by cells.

\section{Cell microscopy, optical structure, and GFP distributions}

When imaging cells, the visible parts of the cell and the optical properties of any edges and colours observed depend on the materials that make up the cell.

Transparency and visible cell edges
Cell edges and interiors
Cell features: protrusions, attachment, cell edge and halo effect
GFP injection and different distributions of GFP inside a cell

\section{Image processing and segmentation}

Edge detection
Blob detection
Watershed and distance transforms: separating objects
3D segmentation

\section{Cellprofiler and segmentation software}

Cellprofiler secondary objects
ImageJ lasso tool
Lack of 3D support

\section{Cell tracking}

The LAP tracking algorithm
Errors caused by long time delay between frames

\section{The Selinummi brightfield profile method}

The 3D brightfield profile
Mathematical properties of the profile
Disadvantages of the Selinummi method
