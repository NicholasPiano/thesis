%*******************************************************************************
%****************************** Fourth Chapter *********************************
%*******************************************************************************

\chapter{Conclusion}

\ifpdf
    \graphicspath{{Chapter6/Figs/Raster/}{Chapter6/Figs/PDF/}{Chapter6/Figs/}}
\else
    \graphicspath{{Chapter6/Figs/Vector/}{Chapter6/Figs/}}
\fi

%********************************** %First Section  **************************************
\section{Summary}

Given the limitations on cell segmentation in 3D image data, the method described in this study performs well in overcoming problems and improves on the previously proposed method by Selinummi et al. by a large margin. It also compares favourably with previous attempts at segmentation using conventional means, such as Z projection of the GFP and segmentation at a constant Z level. It can also be used as a method to correct the autofocus fluctuations in brightfield data, which while not originally intended, is a natural consequence of the 3D data used.

\section{Further work}

Could be used to help locate seeds in images to simplify microscope imaging.
Currently the method only does pre-processing. Can be used instead to do segmentation directly using more information than a current segmentation algorithm is able to use.
