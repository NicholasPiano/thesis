%*******************************************************************************
%****************************** Fourth Chapter *********************************
%*******************************************************************************

\chapter{Method}

\ifpdf
    \graphicspath{{Chapter4/Figs/Raster/}{Chapter4/Figs/PDF/}{Chapter4/Figs/}}
\else
    \graphicspath{{Chapter4/Figs/Vector/}{Chapter4/Figs/}}
\fi

%********************************** %First Section  **************************************
\section{The GFP profile}

This method is based on the Selinummi brightfield method described in the previous chapter. An aim in developing it was to improve on two key problems. The first was the unwanted highlighting caused by objects other than those stained with GFP, allowing for a multicellular environment.

Properties of the vertical GFP distribution
Z position
Variance image compared with absolute GFP intensity

\section{Optimum features for cell recognition}

Dark edges and uniform object interiors
Vertical difference between maximum GFP and ideal features

\section{zMod and zBF}

zMod: use information about Z position of each pixel from profile
zBF: map zMod to brightfield data and sample corresponding pixels into a final 2D image

\section{Artificial edges for segmentation: zDiff and zEdge}

Cellprofiler secondary object segmentation cannot be bounded
Impose boundaries using GFP halo to denote maximum extent of the cell
