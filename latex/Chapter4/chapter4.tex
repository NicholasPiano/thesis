%*******************************************************************************
%****************************** Fourth Chapter *********************************
%*******************************************************************************

\chapter{Experimental apparatus}

\ifpdf
    \graphicspath{{Chapter4/Figs/Raster/}{Chapter4/Figs/PDF/}{Chapter4/Figs/}}
\else
    \graphicspath{{Chapter4/Figs/Vector/}{Chapter4/Figs/}}
\fi

%********************************** %First Section  **************************************
\section{The micro-fluidics environment}

Endothelial cells are the cells that make up the blood vessel wall. This wall is held together by the extra-cellular matrix. As an analogy, if the cells are considered to be a brick wall, the extra-cellular matrix is the mortar. Cancer cells flowing in the bloodstream must break through this barrier in order to metastasize to some other tissue and then grow. [ref] This process can be simulated so that it can be observed more conveniently.

A micro-fluidics device is a 3D environment that can simulate the physical context of a blood vessel. Its substrate is a layer of PDMS (a popular type of plastic used in microscopy studies). Soft lithography is used to imprint a micro-channel framework on the substrate. These micro-channels are roughly the same order of magnitude as a blood vessel.

The refractive index of the PDMS is very similar to that of the oil used in the immersion lens in the confocal microscope. This means that the plastic substrate will not be very visible in the images produced by the confocal microscope, and will not interfere with the observation of the cells.

The micro-channel framework consists of a central channel, which mimics the bloodstream. The central channel is filled with a medium that simulates blood plasma. Side channels radiate away from the central channel. These side channels each represent the wall of a single blood vessel. Collagen gel is placed in the side channels in order to simulate the extra-cellular matrix.

Each side channel is seeded with blood vessel (endothelial) cells, which will link together to form a blood vessel wall across the side channel. The central channel is seeded with cancer cells, which have been marked with fluorescent GFP. We wish to detect how the cancer cells penetrate the blood vessel walls.

The environment is observed via a confocal microscope. An "experiment" in this context is a ~15-hour (between 10 and 20 hours) time period in which a single micro-fluidics device, having been seeded with cells, is observed with a microscope. The micro-fluidics device is mounted on a movable stage. The stage performs a sequence of movements that position each side channel location in turn under the microscope. The stage also adjusts the vertical height of the micro-fluidics device in order to vary the focal plane at which the location is observed.

At any particular location, data is gathered for 1 minute. A laser is used to excite the GFP proteins as data is gathered. The cells then require between 3 and 4 minutes to cool down and radiate away excess energy. During this cooldown period, the microscope is moved between different locations. The microscope scans the locations above the side channels in a repeating sequence. The combined period between imaging the same set of cells is about 10 minutes. This is a very low temporal resolution.

A "series" refers to the data collected from a particular location scanned by the microscope. 5 to 8 series can be produced by one microscope collecting data from one micro-fluidics device.

%********************************** %Second Section  **************************************
\section{Imaging limitations}

We want to observe the cancer cells during the experimental time period in as much detail as possible, so we want to produce images as quickly as possible. However, our experimental apparatus sets some inherent limits on the capture rate and resolution of image data.

Firstly, we are imaging live cells. Cell must be alive and behave naturally during the imaging process. Cells have a maximum rate at which they can absorb energy without dying. [ref] In order to observe cells via a confocal microscope's GFP detector, we use a laser to excite the GFP proteins within the cells. We must therefore introduce a cooldown period after producing each image so that the cells can successfully radiate the energy as heat.

Secondly, we want to use a wide field of view. This allows us to view many cells at the same time and to track a single cell over a large area (it might move around quite a lot during a 15-hour period). However, this requires a lower resolution, reducing the cell detail in the images we produce.

Thirdly, the mechanics of the movable stage have precision limits. The microscope produces an image at a specific location using a specific focal length. After the stage has moved the micro-fluidics device once through its movement sequence, and then returned it to the original position, the real focal length will not be the same. This effect can be somewhat compensated for by using the microscope's autofocus mechanism. The autofocus adjusts the focal length over a range, searching for an image entropy level that matches the entropy level of the previous image it produced at that location. This compensation is not always reliable.

Fourthly, the GFP data has some limitations in terms of detail. We are interested in the overall  cell shape and other properties such as protrusions (which indicate how the cell might be moving) over time. However, images produced using GFP detection are grainy and the cell edges are not clearly defined. The GFP stains cytoplasm, which is not present in the nucleus, so in the GFP data the nucleus is visible only as an empty hole within the cell. Also, the filaments that make up cell protrusions generally have only small amounts of cytoplasm. Protrusions are thus also often not found via GFP detection.

Finally, the best features for cell segmentation in the Brightfield do not always occur in the same focal plane as the most high-quality (most intense) GFP results. The difference in the z-dimension between these two focal planes is specific to a particular microscope during a particular experiment. In this paper, we use the term "delta-z" to denote the very empirical relationship between these two focal lengths.
